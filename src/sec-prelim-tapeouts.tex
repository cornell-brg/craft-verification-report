%=========================================================================
% sec-prelim-tapeouts
%=========================================================================

\section{Preliminary Subsystem Tape-Outs}

UCSD has fabricated three preliminary test chips to verify various
compoments that will be used in the final CERTUS SoC. Even though these
test chips are in older technologies, the corresponding design,
implementation, and testing process still provides invaluable experience
in managing the ASIC toolflow and mitigating physical design issues. Each
of these three preliminary test chips are briefly described below.

\paragraph{Basejump I/O Test Chip}
This test chip is similar in spirit to the first tape-in described in
Section~\ref{sec-full-sys-functional}. It was fabricated in a TSMC 180nm
technology to test basejump I/O including the \TT{bsg\_comm} interface,
\TT{bsg\_tag} interface, and a simple, fully digital clock generator. The
clock generator was configured through the \TT{bsg\_tag} interface. This
test chip verified the strict timing constraints required by the
source-synchronous signaling protocol used in both the \TT{bsg\_comm} and
\TT{bsg\_tag} interfaces. The simple, fully digital clock generator will
likely be included on the CERTUS SoC as a back-up for the more complex
PLL.

\paragraph{Vanilla-5 Ten-Core Test Chip}
This test chip is similar in spirit to the third tape-in described in
Section~\ref{sec-full-sys-functional}. It was fabricated in a TSMC 180nm
technology and included ten Vanilla-5 cores each with their own
instruction and data memories. The cores are interconnected by the mesh
network, and this mesh network is in turn connected to the basejump I/O.

\paragraph{DLDO Test Chip}
This test chip was fabricated in a TSMC 65nm technology and included just
the digital low-drop-out regulator. This chip validated concepts of PD
control to enable stability over large load impedance ranges, and
duty-control to enable wide dynamic load current range. The silicon
results matched the initial simulation results. This validated our
overall approach and helps to reducing risk for the CERTUS SoC.

