%=========================================================================
% sec-full-sys-physical
%=========================================================================

\section{Full-System Physical Verification Strategy}
\label{sec-full-sys-physical}

The final phase of our overall verification strategy involves physical
verification of each ``tape-in'' produced through the previous phase. We
can divide this phase into three sub-phases: front-end verification,
back-end verification, and manufacturing verification.

\smallskip
\paragraph{Front-End Verification}
We are using Synopsys Design Compiler to synthesize the tape-in RTL into
a gate-level netlist. Front-end verification occurs on this
post-synthesis gate-level netlist.

\smallskip
\begin{cbxlist}{1.5em}{0em}{0.5em}

  \item \IT{Functional Gate-Level Simulation:} We first run the
     full-system functional tests (using the end-to-end testing framework
     described in the previous section) on the post-synthesis gate-level
     netlist without any timing information. This kind of testing can
     identify simulation/synthesis mismatch errors, tool configuration
     errors, and errors in the tools themselves.

  \item \IT{Back-Annotated Gate-Level Simulation:} We then run the
     full-system functional tests (again using the end-to-end testing
     framework described in the previous section) on the post-synthesis
     gate-level netlist with back-annotated timing information. These
     netlists may include gate-level models of mixed-signal subsystems
     (e.g., the PLL in a reduced mode of operation). This kind of testing
     can identify setup time and hold time violations, and can also
     partially verify correct integration of the processor/accelerator
     subsystems with the mixed-signal subsystems.

  \item \IT{Logical Equivalence Checking:} We use Cadence Conformal to
     perform formal equivalence checking between the original RTL and the
     post-synthesis gate-level netlist. This kind of testing can identify
     tool configuration errors and errors in the tools themselves.

\end{cbxlist}

\smallskip
\paragraph{Back-End Verification}
We are using Cadence Innovus to place-and-route the post-synthesis
gate-level netlist. We also use Cadence Innovus for floor planning, power
routing, clock routing, and I/O pad placement. The output of this
subphase is the final GDS.

\smallskip
\begin{cbxlist}{1.5em}{0em}{0.5em}

  \item \IT{Functional Gate-Level Simulation:} We run the full-system
     functional tests (using the end-to-end testing framework described
     in the previous section) on the post-pnr gate-level netlist without
     any timing information. These netlists will include the clock tree
     network along with results of gate resizing, additional buffer
     insertion, and local re-synthesis. This kind of testing can identify
     tool configuration errors, and errors in the tool themselves.

  \item \IT{Back-Annotated Gate-Level Simulation:} We then run the
     full-system functional tests (using the end-to-end testing framework
     described in the previous section) on the post-pnr gate-level
     netlist with back-annotated timing information. This kind of testing
     can identify setup time and hold time violations, and can also
     partially verify correct integration of the processor/accelerator
     subsystems with the mixed-signal subsystems.

  \item \IT{Logical Equivalence Checking:} We use Cadence Conformal to
     perform formal equivalence checking between the post-synthesis
     gate-level netlist and the post-pnr gate-level netlist. This kind of
     testing can identify tool configuration errors and errors in the
     tools themselves.

  \item \IT{Connectivity, Antenna, Timing Check:} We use Cadence Innovus
     to do a connectivity check on the final design. This kind of check
     can detect opens, unconnected wires (geometric antennas),
     unconnected pins, loops, partial routing, and unrouted nets. We also
     use Cadence Innovus to identify process antenna effects (PAE) and
     maximum floating area violations. Finally, we use Cadence Innovus to
     do preliminary static timing analysis based on Innovus RC
     extraction.

  \item \IT{Block-Level Design Rule Check:} We use Cadence Innovus to do
     a preliminary DRC check on the metal interconnect to indicate if any
     metal layer DRC rules are violated. This step helps identify
     problems before exporting and running the design through full DRC.
     We also use Mentor Calibre to do a block-level DRC against the
     foundry-provided design rules.

\end{cbxlist}

\smallskip
\paragraph{Manufacturing Verification}
We perform final verification on the GDS produced from the previous
subphase. Once these final verification steps are complete, the GDS is
ready for fabrication.

\smallskip
\begin{cbxlist}{1.5em}{0em}{0.5em}

  \item \IT{Dummy Pre-Checker Design Rule Check:} We use utilities from
     TSMC to insert extra metals and other layers to ensure density and
     manufacturability rules are satisfied. It is also required to pass
     full DRC. TSMC strongly recommends running a subset of the full DRC
     before running the Dummy Fill utilities. This check identifies
     potentially unfriendly layouts in order to ensure that the
     geometries generated by the utilities are legal.

  \item \IT{Full-Chip Design Rule Check:} We use Mentor Calibre to check
     the design against the foundry-provided (TSMC) design rules. In
     addition to the standard DRC rule deck, we also run DRCs on the DRC
     antenna rule deck and bump rule deck. This check ensures proper
     manufacturing and yield.

  \item \IT{Layout vs. Schematic Check:} We use Mentor Calibre to compare
     the post-pnr transistor-level netlist to the netlist extracted from
     the final layout. This check ensures that the final layout matches
     the intended design.

  \item \IT{Timing Signoff:} We use Synopsys Primetime to perform static
     timing analysis based on the extracted parasitics from the final
     layout. This step ensures that the setup times and hold times for
     the design are being met. It also verifies the skew on the clock
     nets and reports clock timing. The timing checks also consider
     crosstalk noise on the nets providing a stricter timing constraint.

  \item \IT{Signal Integrity Signoff:} We use Synopsys Primetime to
     measure cross-capacitance from nearby wires and check for instances
     where signals could interfere with each other. This check helps
     prevent transient errors due to signals switching at the same time.
     We check for different kinds of noise: above-low, above-high,
     below-low and below-high.

  \item \IT{IR Drop Signoff:} We use Cadence Voltus to perform static
     power analysis and static/dynamic rail analysis. The tool provides
     detailed leakage power, internal power, and switching power reports
     of the design. It also analyzes the power distribution grid and
     determines voltage losses due to the resistances in the power grid
     for both static and dynamic operation. It determines whether the
     power grid density needs to be increased to decrease the overall
     resistance to acceptable levels, otherwise transient runtime errors
     can be caused by low voltages.

  \item \IT{Electromigration Signoff:} We use Cadence Voltus to analyze
     potential wire integrity issues due to electromigration. This check
     ensures higher yields and longevity of the chip. This step is more
     important for high-volume, commercial products that require high
     durability.

  \item \IT{Layout Patterning Check and Shape Simulation:} This check is
     strongly recommended by TSMC for manufacturing. It provides the
     layout simulation according to the photo process sensitivity (photo
     energy, depth of focus, and mask error factor), after optical
     proximity correction (OPC) is applied. We can use the results to
     avoid unfriendly designs and hot-spots for manufacturing. It can be
     run on the cell or chip level.

  \item \IT{Virtual Chemical Mechanical Polishing, Thickness Simulation}
     This check is strongly recommended by TSMC for manufacturing. It
     predicts the thickness of the Cu \& IMD across the die and
     identifies any hot-spots due to critical non-uniform metal
     thickness. It allows us to minimize the thickness variation by
     proper dummy filling or metal slotting.

\end{cbxlist}

