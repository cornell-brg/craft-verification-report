%=========================================================================
% sec-mixed-signal
%=========================================================================

\section{Mixed-Signal Subsystem Functional Verification Strategy}

Working with mixed-signal subsystems can be one of the most challenging
aspects of modern SoC design. In this project, we use all digital,
synthesizable mixed-signal subsystems to facilitate rapid design,
implementation, verification, and porting. The digital PLL is based is
based on a first-order $\Delta\Sigma$ frequency to digital converter
(FDC). The FDC output is added to a desired value controlled via an
off-chip SPI interface, then accumulated and filtered. The filter's
output controls the frequency of one of two ring oscillators that cover
different frequency ranges. The DLDO is digital in the sense that the
feedback control is completely digital, and rather than modulating the
conductance of a single PMOS pass transistor in an analog fashion, we use an array of
binary-weighted, digitally-driven PMOS transistors. We use a mix of
system-, register-transfer-, gate-, and transistor-level modeling to
verify both the PLL and DLDO in isolation.

\paragraph{Validating the PLL Using a System-Level Model}
The PLL is a feedback system that must be carefully designed to meet the
target frequency, phase noise, and settling time requirements. A
bit-exact C language simulator has been written to verify that the design
meets these requirements and matches the behavior predicted by
theoretical analysis. The simulator uses an event-driven algorithm to
rapidly simulate the PLL's output phase error as a function of time. It
models each block within the PLL and includes noise sources as well as
quantization effects from the finite bit widths of the internal digital
busses. The program's inputs are the SPI control settings, and its output
is a sequence of output edge times or phase error samples. The output
sequences are used to evaluate settling time and jitter, and to calculate
the output phase noise spectra (the spectra are calculated via subsequent
Matlab processing). The dedicated C-language simulation program was
written to achieve very rapid yet precise simulations, because long
sequences of phase error samples are needed to verify performance.

\paragraph{Testing the PLL RTL and Gate-Level Models}
A Verilog RTL model was created to implement the control logic, and
various other components of the PLL were modeled abstractly using
Verilog-A. Similarly, post-pnr gate-level models can be simulated along
with Verilog-A models for further verification. Verilog mixed-signal was
performed for numerous test vectors and the results were compared to the
corresponding C language simulator results to verify that the Verilog RTL
model matches the planned design.

\paragraph{Testing the PLL Extracted Transistor-Level Models}
As described in the specification document the digital PLL consists of
several digital blocks and a bank of digitally controlled oscillators
(DCOs). Each DCO contains an inverter-based ring oscillator and each of
the ring oscillator's nodes drives the input of a NAND gate. The
parasitic capacitance of the NAND gate input is changed by changing the
state of the NAND gate's other input, thereby enabling digital control of
the ring oscillator's output frequency. Multiple DCOs are necessary to
cover the PLL's required wide output frequency range with enough overlap
to account for process, supply voltage, and temperature (PVT) variations.
For any given output frequency setting, only one DCO is in use. This DCO
is called the active DCO.

Nearly all of the PLL's digital blocks are clocked by edges that are
synchronous with the output of the active DCO, and the rest are
synchronous with reference clock. Hence, all of these blocks can be
synthesized, simulated, and placed and routed with standard digital
tools. The DCOs consist only of standard-cell digital logic, so they too
can be synthesized, and placed and routed with standard digital tools,
but their transistor-level simulation requirements are more involved than
those of the other PLL blocks. The DCO frequencies depend on parasitic
capacitances and are controlled by modulating parasitic capacitances, so
standard digital simulators are not reliable in predicting the DCO
frequency ranges and they are unable to predict DCO phase noise.

Instead the DCOs are simulated by exacting each DCO from a full version
of the PLL layout (to take proximity and shape effects into account) with
some minor modifications to prevent non-DCO blocks from being simulated
(to reduce simulation time), and then running transistor-level SPICE
simulations. The number of DCOs, and the number and strength of inverters
within each DCO are iteratively adjusted, re-synthesized, re-extracted,
and re-simulated until the bank of DCOs has the necessary frequency range
and phase noise performance. Transient simulations are used to predict
frequencies and periodic steady state simulations are used to predict
phase noise performance throughout this iterative process. It is hoped
that this iterative design process can be automated in the future.

\paragraph{Testing the DLDO Custom Cell Transistor-Level Models}
The power switches and comparators are the only part of the mixed-signal
subsystem that do not use standard cells. Fortunately, these blocks are very conventional and are easy to design. In fact, in the case of the power switches they are literally just single PMOS transistors for which their conductance must be simulated. We aim to design a small family of standard cells consisting of various-width PMOS transistors, and simulating their conductance in a similar maner to how standard cell libraries are simulated. Simulation results will be tabulated under various gate, source, and drain conditions over process corners and temperature for use in subsequent non-SPICE simulations, just as in standard cell developement. Additional conductance beyond the small library can be easily attained by adding more cells in parallel. 

The comparator, on the other hand, is only a single block that will be designed and simulated using conventional SPICE-level simulations. Monte Carlo simulation will be used to estimate worst-case offset, input-referred noise, and delay across PVT, which will be used to create a performance look-up table for use in subsequent non-SPICE simulations. 


\paragraph{Testing the DLDO RTL and Gate-Level Models}
Since the digital control is fully synthesized, its performance can be
well described by RTL simulations. As with the PLL, we can test the full
DLDO using a mix of Verilog RTL and VerilogA look-up-table models of the PMOS power transistors and comparator.  With this simulation setup, we can
accurately predict performance metrics such as loop dynamics in response to load
or reference voltage changes. 


\paragraph{Using DLDO Post-Silicon Configuration to Avoid Pre-Silicon Verification}
Our new SAR DLDO architecture with proportional-derivative (PD) compensation and buck-like
duty-control is naturally stable across a wide range of practical load
conditions. Unlike conventional linear-search DLDO architectures, which
have limited load regulation ranges (set by the width of a single finger
of the LDO), and can suffer from instability if the characteristics of
the load changes, the proposed LDO architecture is robust over many
possible load conditions. What this means is that if the load
characteristics change within the design process (say via a change in
specifications), the design comes in a different corner, or the power
switch conductance is different than expected, for example, the DLDO
architecture will still work and perform well. If there are slight
deviations in performance than what is expected, extra redundancy can be
enabled after fabrication to bring the design up to specification. The
ability to provide post-silicon configuration can significantly reduce
pre-silicon verification, especially for sophisticated mixed-signal
subsystems.

