%=========================================================================
% sec-full-sys-functional
%=========================================================================

\section{Full-System Functional Verification Strategy}
\label{sec-full-sys-functional}

Each tape-in is an instance of a complete chip, but which only implements
a subset of the desired final functionality. Each tape-in gradually adds
more complexity until we converge on the final full-system. Each tape-in
includes the basejump I/O and adheres to the pin-out constraints of the
final chip. Each tape-in supports end-to-end testing in simulation of the
BaseJump Daughterboard from the gateway FPGA firmware out to the PCB and
into the ASIC. Each tape-in comes with tests that are loaded into the
gateway FPGA, which then drive packets through the \TT{bsg\_comm}
interface to accurately simulate communication between the gateway FPGA
and ASIC. The exact same testing infrastructure can be used to test an
actual tape-out, and indeed this is a key part of our post-silicon
bring-up strategy for the final CERTUS SoC. Each of the incremental
tape-ins are briefly described below.

\paragraph{Basejump I/O}
Our first tape-in includes just the basejump I/O for loop-back testing.
This tape-in helps test the source-synchronous off-chip communication
interfaces, the on-chip asynchronous interfaces, and the overall
end-to-end testing methodology. This tape-in also enables testing the
\TT{bsg\_tag} interface: a write-only JTAG-like interface that uses 3
pins to set configuration bits on multiple devices in a chain. Later
physical verification of this tape-in enables testing pad-ring
instantiation in our ASIC flow.

\paragraph{Basejump I/O with Single Vanilla-5 Core}
This tape-in includes a single Vanilla-5 core along with the associated
SRAMs for instruction and data memories. Self-checking test programs are
loaded from the gateway FPGA into the Vanilla-5 cores' instruction memory
through the basejump I/O and \TT{bsg\_comm} interface. The Vanilla-5 core
can send test results back to the gateway FPGA also through the basejump
I/O and the \TT{bsg\_comm} interface. Later physical verification of this
tape-in enables testing the use of SRAMs in our ASIC flow.

\paragraph{Basejump I/O with Vanilla-5 Ten-Core}
This tape-in includes ten Vanilla-5 cores interconnected through a simple
mesh network. The mesh network is directly connected to the basejump I/O
and \TT{bsg\_comm} interface. This enables the gateway FPGA to send and
receive messages through the on-chip mesh network. As with the previous
tape-in, self-checking test programs can be loaded from the gateway FPGA
into each of the Vanilla-5 cores' instruction memories, and the Vanilla-5
cores's can send test results back to the gateway FPGA. Later physical
verification of this tape-in enables testing the hierarchical ASIC flow.

\paragraph{Basejump I/O with Single Rocket Core}
This tape-in includes a single rocket core with nothing attached to the
RoCC interface. The back-side of the Rocket instruction and data caches
use the NASTI interface from UC Berkeley. Adapters are used to connect
this interface to the basejump I/O. Memory requests due to cache misses
from the Rocket core are serviced by the gateway FPGA through the
\TT{bsg\_comm} interface. A self-checking test program is loaded into the
gateway FPGA and executed by the Rocket core after reset. The gateway
FPGA can snoop memory write requests from the Rocket core to check the
test results.

\paragraph{Basejump I/O with Rocket+Manycore Composition}
This tape-in includes the single Rocket core from a previous tape-in
composed with the Vanilla-5 manycore. The Rocket core manages the
Vanilla-5 manycore via the RoCC interface. As in the previous tape-in the
back-side of the Rocket instruction and data caches are connected to the
basejump I/O and gateway FPGA. The Vanilla-5 manycore does not have a
direct connection to the basejump I/O. A self-checking test program is
loaded into the gateway FPGA and executed by the Rocket core after reset.
This test program will in turn load instructions and data into the
Vanilla-5 manycore. The Rocket core will monitor the execution of the
Vanilla-5 manycore and write results to main memory. The gateway FPGA can
snoop memory write requests from the Rocket core to check the test
results.

\paragraph{Basejump I/O with Rocket+Accelerator Composition}
This tape-in includes the single Rocket core from a previous tape-in
composed with the BNN accelerator. The Rocket core manages the BNN
accelerator via the RoCC interface. As in the previous tape-in the
back-side of the Rocket instruction and data caches are connected to the
basejump I/O and gateway FPGA. A self-checking test program is loaded
into the gateway FPGA and executed by the Rocket core after reset. This
test program will in turn configure and start the BNN accelerator. Note
that unlike the Rocket+Manycore composition, the BNN accelerator has a
direct connection to the Rocket data cache and will cause its own memory
requests to go out to the gateway FPGA. The Rocket core will monitor the
execution of the BNN accelerator, and ensure all test results are flushed
to the gateway FPGA for verification.

\paragraph{Basejump I/O with Rocket+Manycore and Rocket+Accelerator
 Composition}
This tape-in simply composes the previous two tape-ins. Testing is
identical to the approach used in the previous tape-ins. This tape-in
helps test the ability of the basejump I/O to support multiple clients
and for the gateway FPGA to manage multiple memory request streams from
different on-chip Rocket cores.

\paragraph{Basejump I/O with Single Rocket Core with the PLL}
This tape-in includes the single Rocket core from a previous tape-in with
the on-chip synthesizable PLL. The PLL is configured through a dedicated
off-chip SPI interface and generates the clock for the Rocket core.
Co-simulation of the Rocket RTL and PLL gate-level models is possible
through careful configuration of the PLL to avoid the need for behavior
which cannot be captured in gate-level simulation. Later physical
verification of this tape-in enables testing integrating mixed-signal
clocking subsystems into our ASIC flow.

\paragraph{Basejump I/O with Vanilla-5 Ten-Core with the DLDO}
This tape-in includes the Vanilla-5 ten-core from a previous tape-in with
the on-chip synthesizable DLDO. The DLDO does not really play a role in
RTL verification, so the primary purpose of this tape-in is to enable
later physical verification of integrating mixed-signal power regulation
subsystems into our ASIC flow.

\paragraph{Basejump I/O with Rocket+Manycore, Rocket+Accelerator,
    Vanilla-5 Ten-Core, PLL, DLDO}
This is the final tape-out which composes the previous tape-ins into the
complete CERTUS SoC. The Rocket+manycore, Rocket+accelerator, and the
Vanilla-5 ten-core are connected to the basejump I/O and then the
\TT{bsg\_comm} interface; the mixed-signal DLDO is connected to the
top-level configuration pins and the Vanilla-5 ten-core; and the
mixed-signal PLL is connected to its dedicated off-chip SPI interface and
is used to clock the entire chip. The \TT{bsg\_tag} interface is used
where necessary for out-of-band configuration. For example, the
\TT{bsg\_tag} interface can disable a subsystem (e.g., one of the Rocket
cores) or enable scanning out specific internal buses for post-silicon
debugging and testing.

