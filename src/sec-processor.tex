%=========================================================================
% sec-processor
%=========================================================================

\section{Processor Subsystem Functional RTL Verification Strategy}

The processor subsystem includes larger RISC-V RV64G Rocket cores as well
as smaller RISC-V RV32IM Vanilla-5 cores. The Rocket core includes 32KB
instruction and data caches and can run complex C/C++ applications. The
Vanilla-5 core is much simpler. It includes smaller instruction and data
local memories and can run simple C applications using a remote store
programming model. The Rocket core design uses open-source Chisel-based
IP from UC Berkeley, while the Vanilla-5 core was designed by open source collaborators working with the CERTUS
team.

\paragraph{Testing BSG RTL Component Library}
We are using the open source BSG RTL component library, bsg\_ip\_cores, which was developed for teaching hardware design concepts to UCSD students.
This library includes muxes, counters, arbiters, decoders, encoders,
arithmetic blocks, queues, register files, and memories. Each component
includes unit tests to verify functionality. Because these components are
highly parameterized, we use parameterized testing to sweep over a broad
set of configurations. We include support for regression testing which
runs every unit test to ensure that modifications do not break other
components with dependencies. The BSG RTL component library enables reuse
across other parts of the SoC. For example, approximately 62 of the 65
modules that make up the \TT{bsg\_comm\_link} interface, and approximately 100 of
the 122 modules that make up the Vanilla-5 ten-core come from the BSG IP
core library. 

\paragraph{Testing the On-Chip Network RTL}
We unit test the on-chip mesh network in isolation by using custom test
harnesses. These tests inject various message patterns into network input
terminals and verify the messages are ejected at the correct output
terminal. Additionally, the manycore ISA provides low-level access to the on-chip network, which allows testing of low-level parameters like fairness using short programs.

\paragraph{Testing Single Vanilla-5 Core RTL}
To test a single Vanilla-5 core we use 47 carefully crafted,
self-checking assembly test programs distributed by UC Berkeley as part
of their RISC-V Vscale design. Each assembly test program tests a single
type of assembly instruction using different inter-instruction dependency
conditions, input data designed to trigger corner cases, and random input
data.

\paragraph{Testing the Vanilla-5 Ten-Core/Manycore RTL}
We use small C test programs to test a fully connected ten-core or
manycore design. Whenever a microarchitectural feature is added or bug is
found, we create one of these C programs that stresses that feature or
bug to verify its correctness. For example, we have a test which causes
two data dependencies to occur at the same time while the pipeline is
stalled. The test harness loads the test program into the Vanilla-5
instruction memories, and then waits for the Vanilla-5 manycore to return
the test results.

\paragraph{Testing a Single Rocket Core RTL}
Initial testing of a single Rocket core uses the comprehensive assembly
test suite from UC Berkeley. These assembly test programs test all user-
and supervisor-level instructions. Each test program ensures a given
implementation correctly handles various hazards and corner cases. More
specifically, we use the following RISC-V ``test virtual machines'' and
``test virtual environments'':

\smallskip
\begin{cbxlist}{1.5em}{0em}{0em}
  \item \parbox{0.7in}{rv64ui-p}  64-bit user-level, integer only, vm disabled, only core 0 boots
  \item \parbox{0.7in}{rv64mi-p}  64-bit user-level, mul/div only, vm disabled, only core 0 boots
  \item \parbox{0.7in}{rv64ui-pt} 64-bit user-level, integer only, vm disabled, all cores boot up
  \item \parbox{0.7in}{rv64um-pt} 64-bit user-level, mul/div only, vm disabled, all cores boot up
  \item \parbox{0.7in}{rv64ua-pt} 64-bit user-level, amo only, vm disabled, all cores boot up
  \item \parbox{0.7in}{rv64ui-v}  64-bit user-level, integer only, vm enabled
  \item \parbox{0.7in}{rv64um-v}  64-bit user-level, mul/div only, vm enabled
  \item \parbox{0.7in}{rv64ua-v}  64-bit user-level, amo only, vm enabled
  \item \parbox{0.7in}{rv64uf-p}  64-bit user-level, integer and floating-point, vm disabled, only core 0 boots
  \item \parbox{0.7in}{rv64uf-pt} 64-bit user-level, integer and floating-point, vm disabled, all cores boot up
  \item \parbox{0.7in}{rv64uf-v}  64-bit user-level, integer and floating-point, vm enabled
  \item \parbox{0.7in}{rv64si-v}  64-bit supervisor-level, integer only
\end{cbxlist}
\smallskip

Once we verify the Rocket core using these assembly tests, we then use
small C test programs running on the bare-metal software stack, meaning
they do not use the memory management unit nor any system calls. This
enables testing more complex mixed-instruction interactions. Finally, we
use these same C test programs along with additional, more complex C test
programs running on the proxy-kernel software stack. The proxy-kernel
software stack enables the memory management unit and allows system calls
to marshalled and sent to a host machine for servicing (i.e.,
``tethered'' mode).

\paragraph{Testing the Rocket+Manycore RTL}
We have developed testing infrastructure to verify multiple Rocket cores
composed with a Vanilla-5 manycore through the RoCC interface. The Rocket
core loads instructions and data into the manycore and waits for a
response.

